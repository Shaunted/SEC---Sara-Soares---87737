\section{Instruction Set}
\label{sec:isa}

Versat controller features a very minimal set of instructions to
control the execution hole system where it is integrated. 

There is only one instruction type illustrated in Table~\ref{tab:if}.
\begin{table}[!htbp]
  \centering
    \begin{tabular}{|c|p{7cm}|}
    \hline 
    {\bf Bits} & {\bf Description} \\
    \hline \hline 
     7-4 & Operation code (opcode)\\
    \hline
     3-0 & Immediate constant \\
    \hline

    \end{tabular}
  \caption{Instruction format.}
  \label{tab:if}
\end{table}


The instruction set is given in Table~\ref{tab:isa}. Brackets are used
to represent memory pointers. For example, M[Imm] represents the
contents of the memory position whose address is Imm.


\begin{table}[!htbp]
  \centering
    \begin{tabular}{|c|c|p{8cm}|}
    \hline 
    {\bf Mnemonic} & {\bf Opcode} & {\bf Description} \\
    \hline \hline 
    %nop & 0x0 & No operation; PC = PC+1\\
    %\hline
    rdw   & 0x0 & RA = M[Imm]; PC = PC+1\\
    \hline
    wrw   & 0x1 & M[Imm] = RA; PC = PC+1\\
    \hline
    rdwb  & 0x2 & RA = M[RB+Imm]; PC = PC+1\\
    \hline
    wrwb  & 0x3 & M[RB+Imm] = RA; PC = PC+1\\
    \hline
    beqi  & 0x4 & RA == 0? PC = Imm: PC += 1; RA = RA-1\\
    \hline
    beq   & 0x5 & RA == 0? PC = M[Imm]: PC += 1; RA = RA-1\\
    \hline
    bneqi & 0x6 & RA != 0? PC = Imm: PC += 1; RA = RA-1\\
    \hline
    bneq  & 0x7 & RA != 0? PC = M[Imm]: PC += 1; RA = RA-1\\
    \hline
    ldi   & 0x8 & RA = Imm; PC=PC+1\\
    \hline
    ldih  & 0x9 & RA[7:4] = Imm; PC=PC+1\\
    \hline
    shft  & 0xA & RA = (Imm $<$ 0)? RA $<<$ 1: RA $>>$ 1; PC=PC+1\\
    \hline
    add   & 0xB & RA = RA + M[Imm]; PC=PC+1\\
    \hline
    addi  & 0xC & RA = RA + Imm; PC=PC+1\\
    \hline
    sub   & 0xD & RA = RA - M[Imm]; PC=PC+1\\
    \hline
    and   & 0xE & RA = RA \& M[Imm]; PC=PC+1\\
    \hline
    xor   & 0xF & RA = RA $\oplus$ M[Imm]; PC=PC+1\\
    \hline
    
    \end{tabular}
  \caption{Instruction Set.}
  \label{tab:isa}
\end{table}

\subsection{Virtual Instructions}

There is also a {\tt nop} instruction, which do not have its own
opcode. The assembler will translate this instruction as an {\tt addi
  0}. From the progammer point of view, it is an instruction available
as any other.

\subsection{Delayed Branches}

The PC increments by one after non branch instructions. For branch or
flow control instructions ({\tt beqi, beq, bneqi, bneq}), the PC is
assigned the branch target value, PC + Imm or RB, depending on whether
it is a relative branch ({\tt beqi, bneqi}) or an absolute branch
({\tt beq, bneq}), respectively. Due to the controller pipeline
circuits, the new value of the PC is delayed by 1 instruction. Hence,
the 1 instructions immediately after the branch instruction is
executed and constitute a delay slot. If one lacks a useful
instruction for this slot, one should be filled with {\tt nop}
instruction.

