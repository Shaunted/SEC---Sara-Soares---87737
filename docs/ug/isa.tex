\section{Instructions}
\label{sec:isa}

\subsection{Instruction Format}

PicoVersat features a very minimal set of instructions. There is only one
instruction type as illustrated in Table~\ref{tab:if}.

\begin{table}[!htbp]
  \centering
    \begin{tabular}{|c|p{7cm}|}
    \hline 
    {\bf Bits} & {\bf Description} \\
    \hline \hline 
     31-28 & Operation code (opcode)\\
    \hline
     27-0 & Immediate constant \\
    \hline

    \end{tabular}
  \caption{Instruction format.}
  \label{tab:if}
\end{table}


\subsection{Instruction Set Architecture}

The instruction set is given in Table~\ref{tab:isa}. The notation M[Imm]
represents the contents of position (address) Imm in the memory map.

\begin{table}[!htbp]
  \centering
    \begin{tabular}{|c|c|p{8cm}|}
    \hline 
    {\bf Mnemonic} & {\bf Opcode} & {\bf Description} \\
    \hline \hline 
    %nop & 0x0 & No operation; PC = PC+1\\
    %\hline
% 
\multicolumn{3}{|c|}{{\bf Arithmetic / Logic}}\\
\hline \hline
    addi  & 0x0 & RA = RA + Imm; PC=PC+1\\
    \hline
    add   & 0x1 & RA = RA + M[Imm]; PC=PC+1\\
    \hline
    sub   & 0x2 & RA = RA - M[Imm]; PC=PC+1\\
    \hline
    shft  & 0x3 & RA = (Imm $<$ 0)? RA $<<$ 1: RA $>>$ 1; PC=PC+1\\
    \hline
    and   & 0x4 & RA = RA \& M[Imm]; PC=PC+1\\
    \hline
    xor   & 0x5 & RA = RA $\oplus$ M[Imm]; PC=PC+1\\
    \hline
\multicolumn{3}{|c|}{{\bf Load / Store}}\\
\hline \hline
    ldi   & 0x6 & RA = Imm; PC=PC+1\\
    \hline
    ldih  & 0x7 & RA[31:16] = Imm; PC=PC+1\\
    \hline
    rdw   & 0x8 & RA = M[Imm]; PC = PC+1\\
    \hline
    wrw   & 0x9 & M[Imm] = RA; PC = PC+1\\
    \hline
    rdwb  & 0xA & RA = M[RB+Imm]; PC = PC+1\\
    \hline
    wrwb  & 0xB & M[RB+Imm] = RA; PC = PC+1\\
    \hline
\multicolumn{3}{|c|}{{\bf Branch}}\\
\hline \hline
    beqi  & 0xC & RA == 0? PC = Imm: PC += 1; RA = RA-1\\
    \hline
    beq   & 0xD & RA == 0? PC = M[Imm]: PC += 1; RA = RA-1\\
    \hline
    bneqi & 0xE & RA != 0? PC = Imm: PC += 1; RA = RA-1\\
    \hline
    bneq  & 0xF & RA != 0? PC = M[Imm]: PC += 1; RA = RA-1\\
    \hline
    
    \end{tabular}
  \caption{Instruction Set.}
  \label{tab:isa}
\end{table}

\subsection{Aliased Instructions}

There are two instruction aliases:
\begin{description}
\item {\tt nop}: same as  {\tt addi 0}. It means No OPeration or do nothing.
\item {\tt wrc A,B}: same as {\tt wrw (A+B)}, where A and B are numeric and
  integer constants given with the {\tt wrc} alias.
\end{description}

\subsection{Delayed Branches}

The instruction following a branch instruction is always executed to the
instruction pipeline latency (delayed branch or slot). Make sure instructions
placed after branch instructions are useful and not harmful.

Insert {\tt nop} instructions after a branch if there is no suitable candidate
for the delay slot·

