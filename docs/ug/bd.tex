\section{Block Diagram}

The picoVersat block diagram is shown in Fig.~\ref{fig:bd}.

\begin{figure}[!htbp]
    \centerline{\includegraphics[width=\textwidth]{bd}}
    \vspace{0cm}\caption{Block Diagram}
    \label{fig:bd}
\end{figure}


PicoVersat contains 4 main registers: the accumulator (register A), the data
pointer (register B), the flags register (register C) and the program counter
(register PC).

\subsection{Accumulator register}

Register A, the accumulator, is the main register in this architecture. It can
be loaded with an immediate value from the instruction itself (immediate value)
or with a value read from the data interface. It is the destination of
operations using as operands register A itself and an immediate or addressed
value. Its value is driven out to the data interface.

\subsection{Pointer register}

Register B, the memory pointer, is used to implement indirect loads and
stores to/from the accumulator, respectively. Its contents is the
load/store address for the data interface. Register B itself is in the memory
map so it can be read or written as if accessing the data interface.

\subsection{Flags register}

Register C, the flags register, is used to store three operation flags: the
negative, overflow and carry flags. Register C itself is in the memory map and
it is read only. The flags are set by the controller ALU and can be read by
programs for decision taking.

\subsection{PC register}

The Program Counter (PC) register contains the address of the next instruction
to be fetched from the Program Memory. The PC normally increments to fetch the
next instruction, except for program branch instructions, in which case the PC
register is loaded with the instruction immediate or with the value in register
B, depending on the branch instruction type, direct or indirect, respectively.

