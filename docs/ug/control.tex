\subsection{Controller}

The Versat controller has a 8-bit minimal architecture to support
simple calculations and managing some peripherals. Its accumulator
architecture is shown in Fig.~\ref{fig:control}.

\begin{figure}[!htbp]
    \centerline{\includegraphics[width=\textwidth]{control}}
    \vspace{0cm}\caption{The Controller.}
    \label{fig:control}
\end{figure}

The controller executes machine code instructions. The instruction set
is explained in section~\ref{sec:isa}. The instructions must be stored in
a Program Memory block. The fetched instruction is decoded; the
operation code (opdocde) and immediate value are identified and the
opcode is decoded to select the ALU operation, read or write modes.

The controller sees the other blocks in the system as memory mapped,
and uses the Control bus to access its memory space.

The controller architecture contains 4 main registers: the accumulator
(register A), the data pointer (register B), the carry bit register
(register C) and the program counter (register PC).

\subsubsection{Accumulator}

Register A, the accumulator, is the main register in this
architecture. It can be loaded with a value read from the instruction
itself (immediate value) or a value read from the Control Bus. It can
perfom operations using as arguments its own contents and an immediate
or addressed value. For that it is preceeded by an ALU block to
compute its next value. Finally, its contents can be written to the
Control Bus.

\subsubsection{Data Pointer}

Register B, the data pointer, is used to implement indirect loads and
stores to/from the accumulator, respectively. Its contents is the
load/store address for the Control Bus. Register B is in the memory
map so it can be read or written using the Control Bus. This register
has two times the data width, so it is addressable to the word level
(most or less significant bits).

\subsubsection{Carry bit register}

Register C, the carry bit register, is used to store the carry bit. It
is a read only register.

\subsubsection{Program Counter}

The PC contains the address of the next instruction to fetch from the
Program Memory. The PC normally increments to fetch the next
instruction or, to implement program jumps, it is loaded with an
instruction immediate or with the register B value.

\subsubsection{Control Bus}
\label{sec:rwbus}


The Control Bus signals shown in Fig.~\ref{fig:control} are described
in Table~\ref{tab:rwbus}.

\begin{table}[!htbp]
  \centering
    \begin{tabular}{|p{1.8cm}|c|p{10cm}|}
    \hline 
    {\bf Name} & {\bf Direction} & {\bf Description} \\
    \hline \hline 
     req & OUT & Read or write request.\\
    \hline
     rnw & OUT & Characterizes the request as read if it is 1 or a write if it is 0. \\
    \hline
     address & OUT & Address to be read or written \\
    \hline
     data2read & IN & Data to be read from the Control Bus \\
    \hline
     data2write & OUT & Data to be written to the Control Bus \\
    \hline

    \end{tabular}
  \caption{Control Bus signals as driven by the controller.}
  \label{tab:rwbus}
\end{table}
